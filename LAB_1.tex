\documentclass{article}
\usepackage[utf8]{inputenc}
\usepackage[left=20mm, top=10mm, right=20mm, bottom=10mm, nohead, nofoot]{geometry}
\usepackage{enumitem}
\setlist[itemize]{itemsep=0pt, parsep=0pt, partopsep=0pt, topsep=0pt}
\usepackage{mathtools}
\usepackage{multicol}
\usepackage[russian]{babel}
\setcounter{page}{84}
\usepackage{setspace}
\setstretch{0.9}
\begin{document}

\begin{figure}[h]
    \centering
    \includegraphics[width=350pt]{1.PNG}
    \label{fig: img_2}    
\end{figure}
\begin{center}{ Figure 2. User interface architecture}\end{center}



\begin{figure}[h]
    \centering
    \includegraphics[width=350pt]{2.PNG}
    \label{fig: img_3}
\end{figure}
\begin{center} Figure 3. User action example \end{center}

\newpage
\begin{figure}[t]
    \centering
     \includegraphics[width=520pt]{3.PNG}
    \label{fig: img_4}
\end{figure}
\begin{center}Figure 4. Sequence diagram for updating the user interface based on user interaction\\ \end{center}



\begin{multicols}{2}
 \noindent interface component model as a reaction to one or
another type of interaction.
\par The user interface implemented according to the specified architecture will be generated on the basis of its
model in the knowledge base, adapted to the needs of
users, and dynamically changed depending on the tasks
to be solved.
\begin {center}
\\IV. Conclusion
\end {center}
The paper analyzes the capabilities of computer systems and the level of development of tools for interaction
with them (user interfaces). Based on the analysis, an 
approach to the design of adaptive user interfaces of
intelligent systems is proposed to provide new scenarios
of user interaction with computer systems.
\par To apply this approach, the OSTIS Technology is used,
which allows solving complex problems and whose components integrate with each other and have a synergistic
effect. In turn, the application of ontological approach
based on the semantic model of building adaptive user interfaces within the framework of the OSTIS Technology
allows making intelligent systems practically applicable
for a large number of classes of tasks. The paper refines
and extends the previously proposed semantic model

\end{multicols}
\newpage
\begin{multicols}{2}
 \noindent of adaptive user interfaces of intelligent systems and
presents their proposed architecture.
\begin{center}
Acknowledgment
\end{center}
The authors would like to thank the scientific collectives of the departments of Intelligent Information Technologies of the Belarusian State University of Informatics
and Radioelectronics and Brest State Technical University
for their help and valuable comments.
\begin{center}
References
\end{center}

\begin{itemize}
   \small\renewcommand{\labelitemi}{ [1]}
\item (2024, February) Data-driven ui: unlimited power.
[Online]. Available: https://mobius-piter.ru/en/2018/spb/talks/
v96lokugwe8cwggio8ois/
\renewcommand{\labelitemi}{ [2]}
\item B. A. Myers and M. B. Rosson, “Survey on user interface
programming,” ser. CHI ’92. New York, NY, USA: Association
for Computing Machinery, 1992, p. 195–202. [Online]. Available:
https://doi.org/10.1145/142750.142789

\renewcommand{\labelitemi}{ [3]}
\item M. E. Sadouski, “Semantic models and tools for designing
adaptive user interfaces of intelligent systems,” \textit{Informatics},
vol. 20, no. 3, p. 74–89, Sep. 2023. [Online]. Available:
http://dx.doi.org/10.37661/1816-0301-2023-20-3-74-89

\renewcommand{\labelitemi}{ [4]}
\item S. Abrahão, E. Insfran, A. Sluÿters, and J. Vanderdonckt,
“Model-based intelligent user interface adaptation: challenges
and future directions,” \textit{Software and Systems Modeling}, vol. 20,
no. 5, p. 1335–1349, Jul. 2021. [Online]. Available: http:
//dx.doi.org/10.1007/s10270-021-00909-7

\renewcommand{\labelitemi}{ [5]}
\item A. Wolff, P. Forbrig, A. Dittmar, and D. Reichart, “Linking gui
elements to tasks: Supporting an evolutionary design process,”
vol. 127, pp. 27–34, 01 2005.

\renewcommand{\labelitemi}{ [6]}
\item D. Shunkevich, “Agentno-orientirovannye reshateli zadach
intellektual’nyh sistem [Agent-oriented models, method and
tools of compatible problem solvers development for intelligent
systems],” in \textit{Otkrytye semanticheskie tekhnologii proektirovaniya
intellektual’nykh system [Open semantic technologies for
intelligent systems]}, V. Golenkov, Ed. BSUIR, Minsk, 2018, pp.
119–132.

\renewcommand{\labelitemi}{ [7]}
\item  N. Zotov, “Semantic theory of programs in next-generation intelligent computer systems,” in \textit{Open semantic technologies for
intelligent systems}. BSUIR, Minsk, 2022, pp. 297—-326.

\renewcommand{\labelitemi}{ [8]}
\item V. N. Lukin, A. L. Dzyubenko, and Y. B. Chechikov,
“Approaches to user interface development,” \textit{Program. Comput.
Softw.}, vol. 46, no. 5, p. 316–323, sep 2020. [Online]. Available:
https://doi.org/10.1134/S0361768820050059

\renewcommand{\labelitemi}{ [9]}
\item M. G. Shishaev and V. V. Dikovickij, “Tekhnologiya sinteza
adaptivnyh pol’zovatel’skih interfejsov dlya mul’tipredmetnyh informacionnyh sistem,”\textit{ Trudy Kol’skogo nauchnogo centra RAN},
no. 5 (24), pp. 101–108, 2014.

\renewcommand{\labelitemi}{ [10]}
\item V. Golenkov, Ed., \textit{Tehnologija kompleksnoj podderzhki
zhiznennogo cikla semanticheski sovmestimyh intellektual’nyh
komp’juternyh sistem novogo pokolenija [Technology of complex
life cycle support of semantically compatible intelligent computer
systems of new generation ]}. Bestprint, 2023.

\renewcommand{\labelitemi}{ [11]}
\item M. Orlov, “Control tools for reusable components of intelligent
computer systems of a new generation,” \textit{Open semantic technologies for intelligent systems}, no. 7, pp. 191–206, 2023.

\renewcommand{\labelitemi}{ [12]}
\item K. Bantsevich, “Metasystem of the ostis technology and the
standard of the ostis technology,” in \tetxit{Open semantic technologies for intelligent systems}, ser. Iss. 6, V. Golenkov, Ed. BSUIR,
Minsk, 2022, pp. 357–368.

\renewcommand{\labelitemi}{ [13]}
\item A. Zagorskiy, “Principles for implementing the ecosystem of
next-generation intelligent computer systems,” \tetxit{Open semantic
technologies for intelligent systems}, no. 6, pp. 347–356, 2022.

\renewcommand{\labelitemi}{ [14]}
\item V. Golenkov, N. Guliakina, and D. Shunkevich,\tetxit{ Otkrytaja
tehnologija ontologicheskogo proektirovanija, proizvodstva
i jekspluatacii semanticheski sovmestimyh gibridnyh
intellektual’nyh komp’juternyh sistem [Open technology of
ontological design, production and operation of semantically
compatible hybrid intelligent computer systems]}, V. Golenkov,
Ed. Minsk: Bestprint [Bestprint], 2021.
\end{itemize}

\begin{center}
\textbf{АДАПТИВНЫЕ ПОЛЬЗОВАТЕЛЬСКИЕ\\
ИНТЕРФЕЙСЫ\\
ИНТЕЛЛЕКТУАЛЬНЫХ СИСТЕМ:\\
РАСКРЫТИЕ ПОТЕНЦИАЛА\\
ВЗАИМОДЕЙСТВИЯ\\
"ЧЕЛОВЕК-СИСТЕМА"\\\\}\\
\end{center}
\begin{center}
\large
Садовский М. Е., Насевич П. Е.,\\
Орлов М. К., Жмырко А. В.
\end{center}
В работе проведен анализ возможностей компьютерных систем и уровня развития инструментов взаимодействия с ними (пользовательских интерфейсов). На
основе анализа предложен подход к проектированию
адаптивных пользовательских интерфейсов интеллектуальных систем на основе Технологии OSTIS. Уточнена и расширена предложенная ранее семантическая
модель таких интерфейсов, приведена предлагаемая
архитектура таких систем. Проектируемые на основе
предложенного подхода адаптивные пользовательские
интерфейсы обеспечат новые сценарии взаимодействия пользователей с компьютерными системами.\\
\begin{flushright}
Received 01.04.2024
\end{flushright}

\end{multicols}

\end{document}
